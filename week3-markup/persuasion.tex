\documentclass[12pt, oneside]{article}   	% use "amsart" instead of "article" for AMSLaTeX format
\usepackage{geometry}                		% See geometry.pdf to learn the layout options. There are lots.
\geometry{letterpaper}                   		% ... or a4paper or a5paper or ... 
\usepackage{graphicx}				% Use pdf, png, jpg, or eps§ with pdflatex; use eps in DVI mode	
\usepackage{amssymb}
\title{Persuasion}
\author{Jane Austen}
\date{1819}
\setcounter{secnumdepth}{0}
\begin{document}
\maketitle
\subsection{Chapter I}
Sir Walter Elliot, of Kellynch Hall, in Somersetshire, was a man who,
for his own amusement, never took up any book but the Baronetage; there
he found occupation for an idle hour, and consolation in a distressed
one; there his faculties were roused into admiration and respect, by
contemplating the limited remnant of the earliest patents; there any
unwelcome sensations, arising from domestic affairs changed naturally
into pity and contempt as he turned over the almost endless creations
of the last century; and there, if every other leaf were powerless, he
could read his own history with an interest which never failed. This
was the page at which the favourite volume always opened:

“ELLIOT OF KELLYNCH HALL.

“Walter Elliot, born March 1, 1760, married, July 15, 1784, Elizabeth,
daughter of James Stevenson, Esq. of South Park, in the county of
Gloucester, by which lady (who died 1800) he has issue Elizabeth, born
June 1, 1785; Anne, born August 9, 1787; a still-born son, November 5,
1789; Mary, born November 20, 1791.”

Precisely such had the paragraph originally stood from the printer’s
hands; but Sir Walter had improved it by adding, for the information of
himself and his family, these words, after the date of Mary’s
birth—“Married, December 16, 1810, Charles, son and heir of Charles
Musgrove, Esq. of Uppercross, in the county of Somerset,” and by
inserting most accurately the day of the month on which he had lost his
wife.

Then followed the history and rise of the ancient and respectable
family, in the usual terms; how it had been first settled in Cheshire;
how mentioned in Dugdale, serving the office of high sheriff,
representing a borough in three successive parliaments, exertions of
loyalty, and dignity of baronet, in the first year of Charles II, with
all the Marys and Elizabeths they had married; forming altogether two
handsome duodecimo pages, and concluding with the arms and
motto:--“Principal seat, Kellynch Hall, in the county of Somerset,” and
Sir Walter’s handwriting again in this finale:--

“Heir presumptive, William Walter Elliot, Esq., great grandson of the
second Sir Walter.”

Vanity was the beginning and the end of Sir Walter Elliot’s character;
vanity of person and of situation. He had been remarkably handsome in
his youth; and, at fifty-four, was still a very fine man. Few women
could think more of their personal appearance than he did, nor could
the valet of any new made lord be more delighted with the place he held
in society. He considered the blessing of beauty as inferior only to
the blessing of a baronetcy; and the Sir Walter Elliot, who united
these gifts, was the constant object of his warmest respect and
devotion.

His good looks and his rank had one fair claim on his attachment; since
to them he must have owed a wife of very superior character to any
thing deserved by his own. Lady Elliot had been an excellent woman,
sensible and amiable; whose judgement and conduct, if they might be
pardoned the youthful infatuation which made her Lady Elliot, had never
required indulgence afterwards. She had humoured, or softened, or
concealed his failings, and promoted his real respectability for
seventeen years; and though not the very happiest being in the world
herself, had found enough in her duties, her friends, and her children,
to attach her to life, and make it no matter of indifference to her
when she was called on to quit them. Three girls, the two eldest
sixteen and fourteen, was an awful legacy for a mother to bequeath, an
awful charge rather, to confide to the authority and guidance of a
conceited, silly father. She had, however, one very intimate friend, a
sensible, deserving woman, who had been brought, by strong attachment
to herself, to settle close by her, in the village of Kellynch; and on
her kindness and advice, Lady Elliot mainly relied for the best help
and maintenance of the good principles and instruction which she had
been anxiously giving her daughters.

This friend, and Sir Walter, did not marry, whatever might have been
anticipated on that head by their acquaintance. Thirteen years had
passed away since Lady Elliot’s death, and they were still near
neighbours and intimate friends, and one remained a widower, the other
a widow.

That Lady Russell, of steady age and character, and extremely well
provided for, should have no thought of a second marriage, needs no
apology to the public, which is rather apt to be unreasonably
discontented when a woman \textit{does} marry again, than when she does \textit{not};
but Sir Walter’s continuing in singleness requires explanation. Be it
known then, that Sir Walter, like a good father, (having met with one
or two private disappointments in very unreasonable applications),
prided himself on remaining single for his dear daughters’ sake. For
one daughter, his eldest, he would really have given up any thing,
which he had not been very much tempted to do. Elizabeth had succeeded,
at sixteen, to all that was possible, of her mother’s rights and
consequence; and being very handsome, and very like himself, her
influence had always been great, and they had gone on together most
happily. His two other children were of very inferior value. Mary had
acquired a little artificial importance, by becoming Mrs Charles
Musgrove; but Anne, with an elegance of mind and sweetness of
character, which must have placed her high with any people of real
understanding, was nobody with either father or sister; her word had no
weight, her convenience was always to give was--she was only Anne.

To Lady Russell, indeed, she was a most dear and highly valued
god-daughter, favourite, and friend. Lady Russell loved them all; but
it was only in Anne that she could fancy the mother to revive again.

\end{document}  
